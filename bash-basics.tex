\documentclass{beamer}

\mode<presentation> {

\usetheme{default}

%\usecolortheme{orchid}

}

\usepackage{graphicx} % Allows including images
\usepackage{booktabs} % Allows the use of \toprule, \midrule and \bottomrule in tables

%----------------------------------------------------------------------------------------
%	TITLE PAGE
%----------------------------------------------------------------------------------------

\title[Bash]{Bash - The Basics} % The short title appears at the bottom of every slide, the full title is only on the title page

\author{} % Your name
\institute[] % Your institution as it will appear on the bottom of every slide, may be shorthand to save space
{
 \\ % Your institution for the title page
\medskip
\textit{chris@cjwfuller.com}
}
\date{\today}

\begin{document}

\begin{frame}
\titlepage
\end{frame}

\begin{frame}
\frametitle{Overview}
\tableofcontents
\end{frame}

%------------------------------------------------
\section{Terminals and Shells}
%------------------------------------------------

\begin{frame}
\frametitle{Terminals and Shells}
Terminal emulators  - \textit{``A program that emulates a video terminal within some other display architecture"} - \url{en.wikipedia.org/wiki/Terminal_emulator}
\\~\\
Examples on OS X:
\begin{itemize}
	\item Terminal (preinstalled)
	\item iTerm 2 - \url{iterm2.com}
	\item Terminator
\end{itemize}
~\\
Terminals run shells e.g. Bash, Zsh, Ksh, fish
\end{frame}

%------------------------------------------------

\begin{frame}
\frametitle{Terminals and Shells}
Bash - \textbf{\textit{``GNU Bourne-Again SHell"}}
\\~\\
GNU - \textit{``A Unix-like operating system. That means it is a collection of many programs: applications, libraries, developer tools, even games"} - \url{gnu.org}
\\~\\
Bourne SHell - The Bourne shell (\texttt{sh}).  Developed by Stephen Bourne at Bell Labs in 1977 for Version 7 Unix - \url{en.wikipedia.org/wiki/Bourne_shell}
\\~\\
We can use shells like Bash to run programs
\end{frame}

%------------------------------------------------
\section{Commands}
%------------------------------------------------

\begin{frame}
\frametitle{Commonly used programs}
\texttt{man, ls, cat, cp, mv, rm, rmdir, mkdir, chmod, ln, sudo, !, echo, pwd, diff, history, find, ssh, scp, less, grep}
\\~\\
Open your terminal emulator...
\end{frame}

%------------------------------------------------

\begin{frame}[fragile]
\frametitle{man}
	\texttt{man} - format and display the on-line manual pages
	\begin{verbatim}
	$ man ls
	\end{verbatim}
\end{frame}

%------------------------------------------------

\begin{frame}[fragile]
\frametitle{pwd and ls}
	\texttt{pwd} - return working directory name
	\texttt{ls} - list directory contents	
	\begin{verbatim}
	$ pwd
	$ ls
	$ ls -al
	\end{verbatim}
\end{frame}

%------------------------------------------------

\begin{frame}[fragile]
\frametitle{cat and less}
	\texttt{cat} - concatenate and print files
	\\
	\texttt{less} - read files
	\begin{verbatim}
$ cat README.md
$ less README.md
	\end{verbatim}
\end{frame}

%------------------------------------------------

\begin{frame}[fragile]
\frametitle{mv and copy}
	\texttt{mv} - move files
	\\
	\texttt{cp} - copy files
	\begin{verbatim}
	$ mv source target
	$ cp source target
	\end{verbatim}
\end{frame}

%------------------------------------------------

\begin{frame}[fragile]
\frametitle{rm, rmdir, mkdir}
	\texttt{rm} - remove directory entries
	\\
	\texttt{rmdir} - remove directories
	\\
	\texttt{mkdir} - make directories
	\begin{verbatim}
	$ rm file
	$ rmdir directory
	$ mkdir directory
	\end{verbatim}
\end{frame}

%------------------------------------------------

\begin{frame}[fragile]
\frametitle{sudo}
	\texttt{sudo} - execute a command as another user (or, ``Shut Up and DO")
	\begin{verbatim}
	$ touch /etc/foo.txt
	$ sudo touch /etc/foo.txt
	$ rm /etc/foo.txt
	$ sudo rm /etc/foo.txt
	\end{verbatim}
\end{frame}

%------------------------------------------------
\section{Pipes}
%------------------------------------------------

\begin{frame}[fragile]
\frametitle{Pipes}
	\begin{verbatim}
	\end{verbatim}
\end{frame}

\end{document} 